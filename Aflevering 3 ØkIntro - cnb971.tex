\documentclass[a4paper, 12pt]{article}
\usepackage{datetime}
\usepackage{advdate}
\usepackage{lipsum}
\usepackage{booktabs}
\setlength{\columnsep}{25 pt}
\usepackage{tabularx,ragged2e,booktabs,caption}
\usepackage{subfigure}
\usepackage{multicol,tabularx,capt-of}
\usepackage{hhline}
\usepackage{multirow}
\usepackage[utf8]{inputenc}
\usepackage{hyperref}
\usepackage{blkarray}
\usepackage[top=3.4cm,left=2.4cm,right=2.4cm,bottom=3.4cm]{geometry}
\usepackage{amsmath}
\usepackage{amssymb}
\usepackage{placeins}
\usepackage{graphicx}
\usepackage{mathrsfs}
\usepackage{listings}
\usepackage{semantic}
\usepackage{pdfpages}
\usepackage{relsize}
%Alt der er procenttegn foran er kommentarer

%Koden nedenfor er til, hvis man vil indsætte et billede
%\FloatBarrier
%\begin{center}
%	\begin{figure}[!ht]	
%		\centering	
%\includegraphics[width=0.4\textwidth]{}
%		\caption{}
%	\end{figure}
%\end{center}
%\FloatBarrier
\usepackage{fancyhdr}


\lhead{Jeppe Færch} %Disse tre er til sidehovedet
\chead{Introduktion til økonomi}
\rhead{18-12-2020}


\pagestyle{fancy}


\begin{document} 
%%%%%%%%%%%%%%%%%%%%%%%%%%%%%%%%%%%%%%%%%%%%%%%%%%%
%%%%%%%%%%%  Dokumentet starter her %%%%%%%%%%%%%%%
%%%%%%%%%%%%%%%%%%%%%%%%%%%%%%%%%%%%%%%%%%%%%%%%%%%



\subsection*{Opgave 1}
\subsubsection*{1.1}
Virksomhedens produktionsfunktion $y = x_{1}^{\alpha}x_{2}^{1- \alpha}$ er en Cobb-Douglas funktion, så den er monoton og konveks, hermed kan vi opstille virksomhedens problem:
$$min_{x_{1}x_{2}} : w_{1}x_{1} + w_{2}x_{2} - \lambda[x_{1}^{\alpha}x_{2}^{1- \alpha} -y]$$
\\
hvilket giver førsteordensbetingelser:
$$w_{1} =  \alpha x_{1}^{\alpha -1} x_{2}^{1- \alpha} =\alpha \left( \dfrac{x_{2}}{x_{1}} \right)^{1 - \alpha}$$
$$w_{2} =  (1- \alpha)x_{1}^{\alpha}x_{2}^{- \alpha} = (1 - \alpha) \left( \dfrac{x_{1}}{x_{2}} \right)^{\alpha} $$
$$\dfrac{w_{1}}{w_{2}} = \dfrac{\alpha}{1 - \alpha} \dfrac{x_{2}}{x_{1}} \Rightarrow \dfrac{x_{2}}{x_{1}}  = \dfrac{w_{1}}{w_{2}} \dfrac{1 - \alpha}{\alpha} \Rightarrow \dfrac{x_{1}}{x_{2}} = \dfrac{w_{2}}{w_{1}} \dfrac{\alpha}{1 - \alpha}$$
\\
For fast $\overline{y}$ indsættes $w_{1},w_{2}$ og $x_{1},x_{2}$ findes:
$$\overline{y} = x_{1}^{\alpha}x_{2}^{1 - \alpha} = \dfrac{x_{1}^{\alpha}}{x_{2}^{\alpha}} \cdot x_{2} = \left( \dfrac{x_{1}}{x_{2}} \right)^{\alpha} x_{2} = \left( \dfrac{w_{2}}{w_{1}} \dfrac{\alpha}{1 - \alpha} \right)^{\alpha} x_{2} \Leftrightarrow$$
$$x_{1} = y \cdot  \left( \dfrac{w_{2}}{w_{1}} \dfrac{\alpha}{1 - \alpha} \right)^{1 - \alpha}$$
$$x_{2} = y \cdot  \left( \dfrac{w_{1}}{w_{2}} \dfrac{1 - \alpha}{\alpha} \right)^{\alpha}$$
\\
Hermed kan vi udlede virksomhedens efterspørgselsfunktioner for $x_{1}$ og $x_{2}$ som funktion af $w_{1},w_{2}$ og $y$.
$$x_{1}^{*}(w_{1},w_{2},y) = y \cdot  \left( \dfrac{w_{2}}{w_{1}} \right)^{1 - \alpha} \left( \dfrac{\alpha}{1 - \alpha} \right)^{1 - \alpha}$$
$$x_{2}^{*}(w_{1},w_{2},y) = y \cdot  \left( \dfrac{w_{1}}{w_{2}} \right)^{\alpha}  \left( \dfrac{1 - \alpha}{\alpha} \right)^{\alpha}$$


%%%%%%%%%%%%%%%%%%%%%%%%%%%%%%%%%%%%%%%%%%%%%%%%%%%%%%%%
\subsubsection*{1.2}
Vi skal finde elasticiteten af $(x_{2}^{*} / x_{1}^{*})$ mht. $(w_{2} / w_{1})$: $\dfrac{\partial(x_{2}^{*} / x_{1}^{*})}{\partial(w_{2} / w_{1})} \cdot \dfrac{(w_{2}/w_{1})}{(x_{2}^{*} / x_{1}^{*})}$
\\
Først finder vi: 
$$\dfrac{\partial(x_{2}^{*} / x_{1}^{*})}{\partial(w_{2} / w_{1})} = \dfrac{\partial \left( \dfrac{(1- \alpha)}{\alpha} \dfrac{w_{1}}{w_{2}}\right)}{\partial(w_{2} / w_{1})} = \dfrac{\partial \left( \left( \dfrac{\alpha}{1 - \alpha} \right)^{-1} \left( \dfrac{w_{2}}{w_{1}} \right)^{-1} \right)}{\partial(w_{2} / w_{1})} = (-1) \cdot \left( \dfrac{\alpha}{1 - \alpha} \right)^{-1} \left( \dfrac{w_{2}}{w_{1}} \right)^{-2}$$
\\
Herefter
$$ \dfrac{(w_{2}/w_{1})}{(x_{2}^{*} / x_{1}^{*})} = \dfrac{\left( \dfrac{w_{2}}{w_{1}} \right)}{\left( \dfrac{\alpha}{1 - \alpha} \right)^{-1} \left( \dfrac{w_{2}}{w_{1}} \right)^{-1}} = \left( \dfrac{w_{2}}{w_{1}} \right)^{2} \left( \dfrac{\alpha}{1 - \alpha} \right)^{1} $$
\\
Hvilket giver elasticiteten:
$$\dfrac{\partial(x_{2}^{*} / x_{1}^{*})}{\partial(w_{2} / w_{1})} \cdot \dfrac{(w_{2}/w_{1})}{(x_{2}^{*} / x_{1}^{*})} = (-1) \cdot \left( \dfrac{\alpha}{1 - \alpha} \right)^{-1} \left( \dfrac{w_{2}}{w_{1}} \right)^{-2} \cdot \left( \dfrac{w_{2}}{w_{1}} \right)^{2} \left( \dfrac{\alpha}{1 - \alpha} \right)^{1} = (-1)$$
\\
Den absolutte værdi af elasticiteten er hermed $|-1| = 1$.

%%%%%%%%%%%%%%%%%%%%%%%%%%%%%%%%%%%%%%%%%%%%%%%%%%%%%%%%
\subsubsection*{1.3}
Eftersom at elasticiteten er 1, så vil dette medføre at efterspørgslen er enhedselastisk. Hvilket betyder at når prisen stiger på en vare stiger med 1 $\%$ så vil mængden falde med et procent. Der vil ikke forekomme ændringer i omkostninger og indtægter. Hermed vil totalomkostningerne for input 1 ikke ændrer sig og forblive konstante.

\subsubsection*{1.4}
Vi skal udlede virksomhedens omkostningsfunktion.
\\
Vi ved at en omkostningfunktion er givet ved følgende: $C= w_{1}x_{1} + w_{2}x_{2}$. Med $x_{1}^{*},x_{2}^{*}$ får vi:
$$C(w_{1},w_{2},y) = w_{1} \cdot x_{1}^{*}(w_{1},w_{2},y) + w_{2} \cdot x_{2}^{*}(w_{1},w_{2},y)$$
\\
Vi indsætter virksomhedens input efterspørgslesfunktioner og får:
$$C(w_{1},w_{2},y) = w_{1} \cdot \left(  y \cdot  \left( \dfrac{w_{2}}{w_{1}} \right)^{1 - \alpha} \left( \dfrac{\alpha}{1 - \alpha} \right)^{1 - \alpha}  \right) + w_{2} \cdot \left(  y \cdot  \left( \dfrac{w_{1}}{w_{2}} \right)^{\alpha}  \left( \dfrac{1 - \alpha}{\alpha} \right)^{\alpha}  \right)$$

$$= y \cdot \left(  w_{1} \left( \dfrac{w_{2}}{w_{1}} \cdot  \dfrac{\alpha}{1 - \alpha} \right)^{1 - \alpha}  + w_{2} \cdot  \left( \dfrac{w_{1}}{w_{2}} \cdot \dfrac{1 - \alpha}{\alpha} \right)^{\alpha}  \right) $$

$$= y \cdot \left(  w_{2}^{1 - \alpha} \cdot \left( \dfrac{\alpha}{1 - \alpha} \right)^{1 - \alpha}  + w_{1}^{\alpha} \cdot \left( \dfrac{\alpha}{1 - \alpha} \right)^{\alpha} \right) $$

$$= \left( \left( \dfrac{\alpha}{1 - \alpha} \right)^{1 - \alpha} +  \left( \dfrac{\alpha}{1 - \alpha} \right)^{\alpha} \right) y \cdot w_{2}^{1 - \alpha} \cdot  w_{1}^{\alpha} $$
\\
Hermed har vi virksomhedens omkostningsfunktion:
$$C(w_{1},w_{2},y) =  \left( \left( \dfrac{\alpha}{1 - \alpha} \right)^{1 - \alpha} +  \left( \dfrac{\alpha}{1 - \alpha} \right)^{\alpha} \right) w_{1}^{\alpha} w_{2}^{1 - \alpha} y $$


%%%%%%%%%%%%%%%%%%%%%%%%%%%%%%%%%%%%%%%%%%%%%%%%%%%%%%%%
\subsubsection*{1.5}
Vi finder virksomhedens marginalomkostninger ved at differentiere omkostningsfunktionen i forhold til produktionen $y$:
$$MC = \dfrac{\partial C(w_{1},w_{2},y)}{\partial y} = w_{1} \cdot  \left( \dfrac{w_{2}}{w_{1}} \right)^{1 - \alpha} \left( \dfrac{\alpha}{1 - \alpha} \right)^{1 - \alpha}   + w_{2} \cdot \left( \dfrac{w_{1}}{w_{2}} \right)^{\alpha}  \left( \dfrac{1 - \alpha}{\alpha} \right)^{\alpha}  $$
\\
Marginalomkostningerne er hermed konstante, eftersom de ikke afhænger af $y$. En ændring i produktionen vil derfor ikke medføre ændringer i marginalomkostningerne. Dette er fordi Cobb-Douglas produktionsfunktionen bruger konstante omkostninger for $w_{1}$ og $w_{2}$, hvor de afhænger af $\alpha$. De marginalomkostninger afhænger derfor ikke af produktionen.

%%%%%%%%%%%%%%%%%%%%%%%%%%%%%%%%%%%%%%%%%%%%%%%%%%%%%%%%%%%%%%%%%%%%%%%%%%%%%%%%%%%%%%%%%%%%%%%%%%%%%%%%%%%%%%%%%%%%%%%%%%%%%%%%%%%%%%%%%%%%%%%%%%%%%%%%%%%%%%%%%%%%%%%%

\subsection*{Opgave 2}

\subsubsection*{2.1}
Givet omkostningsfunktionen $C(y) = F + \frac{\alpha}{1+ \alpha} y^{\frac{1 + \alpha}{\alpha}}$ kan vi finde den enkelte virksomheds udbudsfunktion ved at løse maksimeringsproblemet: $max_{y}: py -C(y)$.
\\
Førsteordensbetingelse: $p - \frac{\partial C(y)}{\partial y} = 0 \Leftrightarrow p - y^{1/ \alpha} = 0 \Leftrightarrow p = y^{1/ \alpha}$.
\\
Vi isolerer $y$ og får den enkelte virksomheds udbudsfunktion $y(p)$:
$$ p = y^{1/ \alpha} \Leftrightarrow y(p) = p^{\alpha}$$

%%%%%%%%%%%%%%%%%%%%%%%%%%%%%%%%%%%%%%%%%%%%%%%%%%%%%%%%
\subsubsection*{2.2}
Vi kan nu udlede den enkelte virksomheds udbudselasticitetet gennem udbudsfunktionen. Udbudselasticiteten er:
$$\dfrac{\partial y}{\partial p} \dfrac{p}{y} = \alpha p^{\alpha - a} \cdot \dfrac{y^{1/ \alpha}}{p^{\alpha}} = \dfrac{\alpha y^{1/ \alpha}}{p} = \dfrac{\alpha y^{1/ \alpha}}{y^{1/ \alpha}} = \alpha$$
Det ses at udbudselasticiteten afhænger fuldkommen af $\alpha$, hvor $\alpha \rightarrow \infty $ går udbudselasticiteten mod uendelig, og for $\alpha \rightarrow 0$ går elasticiteten mod 0.
\\
Udbudselasticiteten er defineret som den procentvise ændring i den udbudte mængde pr. den procentvise ændring i prisen. Givet omkostningsfunktionen $C(y)$ hvor $y=p^{\alpha}$, så vil der ved en givet prisændring forekomme ændringer i udbuddet. Hermed afhænger udbudsfunktionen og udbudselasticiteten af karakteristika fra omkostningsfunktionen.

%%%%%%%%%%%%%%%%%%%%%%%%%%%%%%%%%%%%%%%%%%%%%%%%%%%%%%%%
\subsection*{2.3}
Der er $n$ virksomheder i et marked, der alle tager prisen for givet, og vi skal finde industriens samlede udbud $Y$. De har alle omkostningsfunktionen $C(y)$ defineret tidligere og hermed har alle virksomhederne samme udbudsfunktion $y(p) = p^{\alpha}$. Industriens samlede udbud er lig summen af de enkelte virksomheders udbudsfunktioner. Hermed får vi industriens udbud givet ved:
\begin{equation}
\ Y = \mathlarger{\mathlarger{\sum}}_{i = 1}^{n} y_{i}(p) = \mathlarger{\mathlarger{\sum}}_{i = 1}^{n} p^{\alpha} = n\cdot p^{\alpha}
\end{equation}

%%%%%%%%%%%%%%%%%%%%%%%%%%%%%%%%%%%%%%%%%%%%%%%%%%%%%%%%
\subsection*{2.4}
Markedsligevægten er hvor industriens samlede udbud er lig efterspørgslen i markedet, af den givet ligevægtsprisen $p^{*}$, hvilket kan skrives: $Y(p^{*}) = D(p^{*})$. Vi indsætter vores $Y(p)$, $D(p)$ og isolerer ligevægtsprisen $p^{*}$.
$$Y(p) = D(p) \Leftrightarrow n\cdot p^{\alpha} = \frac{A}{p^{\epsilon}} \Leftrightarrow n\cdot p^{\alpha} \cdot p^{\epsilon} = A \Leftrightarrow p^{\alpha + \epsilon} = \frac{A}{n} \Leftrightarrow p^{*} = \left( \frac{A}{n} \right)^{\frac{1}{\alpha + \epsilon}}$$
\\
Vi indsætter ligevægtsprisen $p^{*}$ i $Y(p^{*})$ og får mængden $Y^{*}$:
$$Y^{*} = Y(p^{*}) = n \cdot (p^{*})^{\alpha} = n \cdot \left( \left( \frac{A}{n} \right)^{\frac{1}{\alpha + \epsilon}} \right)^{\alpha} = n\cdot \left( \frac{A}{n} \right)^{\frac{\alpha}{\alpha + \epsilon}}$$

%%%%%%%%%%%%%%%%%%%%%%%%%%%%%%%%%%%%%%%%%%%%%%%%%%%%%%%%
\subsection*{2.5}
Vi finder elasticiteten af ligevægtsprisen mht. $A$ givet ved $\dfrac{\partial p^{*}}{\partial A}\dfrac{A}{p^{*}}$.
\\
Først finder vi $\dfrac{\partial p^{*}}{\partial A}$ med kædereglen hvor $u=\dfrac{A}{n}$:
$$\dfrac{\partial p^{*}}{\partial A} = \dfrac{\partial u^{\frac{1}{\alpha + \epsilon}}}{\partial u} \dfrac{\partial u}{\partial A} = \dfrac{\frac{A}{n}^{-1+1/  (\alpha + \epsilon)}}{\alpha + \epsilon} \cdot  \dfrac{\partial u}{\partial A} = \dfrac{\frac{A}{n}^{-1+1/  (\alpha + \epsilon)}}{\alpha + \epsilon} \cdot  \dfrac{\frac{\partial}{\partial A} (A)}{n} = \dfrac{\frac{A}{n}^{-1+1/  (\alpha + \epsilon)}}{\alpha + \epsilon} \cdot  \dfrac{1}{n} = \dfrac{\frac{A}{n}^{-1+1/  (\alpha + \epsilon)}}{(\alpha + \epsilon)n} \Leftrightarrow $$ 
$$\dfrac{\partial p^{*}}{\partial A} = \dfrac{\left( \frac{A}{n} \right)^{\frac{1}{\alpha + \epsilon}}}{(\alpha + \epsilon)A} $$
\\
Hermed kan vi finde elasticiteten af ligevægtsprisen:
$$\dfrac{\partial p^{*}}{\partial A}\dfrac{A}{p^{*}} =\left( \dfrac{\left( \frac{A}{n} \right)^{\frac{1}{\alpha + \epsilon}}}{(\alpha + \epsilon)A} \right) \cdot \left( \dfrac{A}{ \left( \frac{A}{n} \right)^{\frac{1}{\alpha + \epsilon}}} \right) = \dfrac{A}{(\alpha + \epsilon )A} = \dfrac{1}{\alpha + \epsilon}$$

%%%%%%%%%%%%%%%%%%%%%%%%%%%%%%%%%%%%%%%%%%%%%%%%%%%%%%%%
\subsection*{2.6}
Givet $\alpha \rightarrow \infty$, så vil elasticiteten gå mod $0$. Dette er fordi nævneren bliver størrere og tælleren forbliver uændret, hvilket vil give et mindre resultat. Givet $\alpha \rightarrow 0$, så vil elasticiteten gå mod $\frac{1}{\epsilon}$.
\\\\
Økonomisk betyder dette at jo højere en $\alpha$, så vil priselasticiteten blive mere uelastisk og efterspørgslen afhænger mindre af potentielle prisændringer.
\\\\
Hvis $\alpha$ går mod $0$, så vil efterspørgslen afhænge af værdien $\epsilon$. Hvis $\epsilon > 1$, så vil priselasticiteten igen blive uelastisk. Givet $\epsilon < 1$ vil elasticiteten blive større end 1 og efterspørgslen vil heraf afhænge mere af prisen på varen, hvor prisstigninger vil føre til mindre efterspørgsel og et fald i revenue. Hermed vil en højere $\epsilon$ for $\alpha \rightarrow 0$ føre til en øget elasticitet.

%%%%%%%%%%%%%%%%%%%%%%%%%%%%%%%%%%%%%%%%%%%%%%%%%%%%%%%%
\subsection*{2.7}
Givet et marked med fuldkommen konkurrence på lang sigt, så vil virksomheder producere til mindst mulige omkostninger som muligt, dette er repræsenteret hvor $AC(y)$ har sit minimum. Vi finder $AC(y)$:
$$AC(y) = \frac{c(y)}{y} = \dfrac{F + \frac{\alpha}{1 + \alpha} y^{\frac{1+ \alpha}{\alpha}}}{y} = \dfrac{F}{y} + \dfrac{\alpha y^{\frac{1}{\alpha}}}{1+ \alpha}$$
\\
Minimum for $AC(y)$ er  hvor dens afledte er lig $0$. Vi finder dens afledte:
$$\dfrac{\partial AC(y)}{\partial y} = -  \dfrac{F}{y^{2}} + \dfrac{ y^{\frac{1}{\alpha}}}{(1+ \alpha) y} = -  \dfrac{F}{y^{2}} + \dfrac{y^{\frac{1}{\alpha}-1}}{1 + \alpha} $$
\\
Den sættes lig $0$ og y isoleres:
$$ \dfrac{\partial AC(y)}{\partial y} = 0 \Leftrightarrow  -  \dfrac{F}{y^{2}} + \dfrac{y^{\frac{1}{\alpha}-1}}{1 + \alpha} = 0 \Leftrightarrow \dfrac{y^{\frac{1}{\alpha}-1}}{1 + \alpha} = \dfrac{F}{y^{2}} \Leftrightarrow y^{\frac{1}{\alpha} - 1} = \dfrac{F}{y^{2}} \cdot (1+\alpha) \Leftrightarrow$$
$$ y^{2}\cdot y^{\frac{1}{\alpha} - 1}  = F(1 + \alpha) \Leftrightarrow y^{\frac{1}{\alpha}+1} = F(1 + \alpha) \Leftrightarrow y = (F(1 + \alpha) )^{\dfrac{1}{\dfrac{1}{\alpha} +1}} = (F(1 + \alpha) )^{\dfrac{\alpha}{1 + \alpha}}$$
\\
Vi kan nu finde prisen på lang sigt. Under fuldkommen konkurrence er $p^{LS} = MC(y^{*}) = AC(y^{*})$. Her bruger vi bruger vi optimum fra 2.1: $p^{LS} = MC(y^{*}) = y^{* 1/ \alpha}$. Hermed er ligevægtsprisen på lang sigt:
$$p^{LS} = \left(  (F(1 + \alpha) )^{\dfrac{\alpha}{1 + \alpha}} \right)^{1 / \alpha} =  (F(1 + \alpha) )^{\dfrac{1}{1 + \alpha}}$$

%%%%%%%%%%%%%%%%%%%%%%%%%%%%%%%%%%%%%%%%%%%%%%%%%%%%%%%%
\subsection*{2.8}
Vi finder elasticiteten af $p^{LS}$ mht. $A$ ved $\dfrac{\partial p^{LS}}{\partial A} \dfrac{A}{p^{LS}}$:
$$\dfrac{\partial p^{LS}}{\partial A} \dfrac{A}{p^{LS}} = 0 \cdot \dfrac{A}{p^{LS}} = 0$$
\\
Her ses det at elasticiteten på lang sigt mht. til $A$ er lig $0$ og dermed konstant. Med en konstant elasticitet, så står vi overfor et andet resultat end det givet i 2.5 for elasticiteten på kort sigt $\dfrac{\partial p^{*}}{\partial A}\dfrac{A}{p^{*}} = \dfrac{1}{\alpha + \epsilon}$. På kort sigt vil ligevægtsprisen ændre sig i forhold til værdier for $\alpha$ og $\epsilon$, men på lang sigt forbliver ligevægtsprisen konstant, hvilket betyder at priselasticiteten er fuldkommen uelastisk.


%%%%%%%%%%%%%%%%%%%%%%%%%%%%%%%%%%%%%%%%%%%%%%%%%%%%%%%%
\subsection*{2.9}
Vi skal opstille efterspørgselkurven og udbudskurven på kort sigt, samt udbudskurven på lang sigt.
\\
På kort sigt har vi efterspørgselskurven opgivet i opgave beskrivelsen: $D(p)=\dfrac{A}{p^{\epsilon}}$, hvor $A > 0, \epsilon > 1$ ellers ville efterspørgselskurven ikke være konveks.
\\
Vi finder udbudskurven på kort sigt ved at isolere $p$ i industriens samlede udbud $Y$:
$$Y = np^{\alpha} \Leftrightarrow p^{\alpha} = \dfrac{Y}{n} \Leftrightarrow S(p^{KS}) = \left( \dfrac{Y}{n} \right)^{1 / \alpha}$$
\\
Se figur 1 for efterspørgselskurven og udbudskurven på kort sigt.\\
\FloatBarrier
\begin{center}
	\begin{figure}[!ht]	
		\centering	
\includegraphics[width=0.8\textwidth]{kort sigt.png}
		\caption{}
	\end{figure}
\end{center}
\FloatBarrier

På lang sigt kan virksomheder vælge hvornår de vil komme ind på markedet og her vil udbudskurven være givet ved minimumspunktet for $AC$, hvilket var givet ved:
$$p = (F(1+ \alpha))^{\dfrac{1}{1 + \alpha}}$$
\\
Se figur 2 for efterspørgselskurven, udbudskurven og markedsligevægten på lang sigt.
\FloatBarrier
\begin{center}
	\begin{figure}[!ht]	
		\centering	
\includegraphics[width=0.8\textwidth]{lang sigt.png}
		\caption{}
	\end{figure}
\end{center}
\FloatBarrier


%%%%%%%%%%%%%%%%%%%%%%%%%%%%%%%%%%%%%%%%%%%%%%%%%%%
%%%%%%%%%%%  Dokumentet slutter her %%%%%%%%%%%%%%%
%%%%%%%%%%%%%%%%%%%%%%%%%%%%%%%%%%%%%%%%%%%%%%%%%%%
\end{document}